%!TEX root =  main.tex

\chapter[Introduction]{Introduction}

\section{Access Management}

What is Access Management in IT? 

In the area of information technology the subject "Access Management" has several subareas

\begin{itemize}
    \item Application Security: restrict users and programs
    \item User Productivity: providing users with ease access to their needs
    \item Middleware technology
\end{itemize}


\section{Authentication}

\emph{Authentication} is the act of confirming the truth of an attribute of a single piece of data (a datum) claimed true by an entity. In contrast with \emph{identification} which refers to the act of stating or otherwise indicating a claim purportedly attesting to a person or thing's identity, authentication is the process of actually confirming that identity.

A good comparison to illustrate the main characteristics of authentication is \emph{border control}. That one you'll most often find on airports or on the actual boarder between countries\footnote{There was a time you had to leave \emph{Schengen Area} to find this seldom experience, nowadays there seems to be an inflation of borderline checks.}.

You leave one domain, or more important: you enter another domain. A domain that has a certain definition, borders, and authorities that execute policies (for good and for bad, however).  Those brave officers want you to identify yourself - your job is to provide them data, exact and distinct data and some prove that you tell the truth.

The data is usually good enough to identify you among all human beings. That quality of your prove must satisfy the procedure of those border checks. With a valid passport you've own a document from a common trusted authority that covers both topics:

\begin{enumerate}
    \item[-] identifying you among all others and (identification)
    \item[-] prove at the same time correctness of this statement (authentication)
\end{enumerate}

For you, as the traveler, the desired outcome of this procedure is your valid entry without much delay.


\emph{Thinking more general - what other situations or outcomes are possible?}

\begin{itemize}
    \item[-] you could pretend to be another person, perhaps from another country
    \item[-] you could sneak in the country, preventing any check
    \item[-] you could mix you passport by mistake in the cafeteria next to the borderline. 
        You enter the country with a different identity without knowing
    \item[-] your passport is expired and you are not allowed to enter the country
    \item[-] etc.
\end{itemize}

All those situations have a direct mapping to use cases in the IT world.

One more thing is in common: it is not checked what exactly you are allowed to do (in the country or domain). There might be a rough categorization by authentication, like people from country X would also need a visa and people from country Y are not allowed at all to enter. But this is not the place where a work or residence permit would be checked (as far as I know).

In IT those procedures were reused.


\section{Authentication Factors}

The ways in which someone may be authenticated fall into three categories, based on what are known as the factors of authentication: something the user \{emph}knows, something the user \emph{has}, and something the user \emph{is}.

Each authentication factor covers a range of elements used to authenticate or verify a person's identity prior to being granted access, approving a transaction request, signing a document or other work product, granting authority to others, and establishing a chain of authority.

For autentication to well-secured systems, elements from at least two, and preferably all three, factors should be verified.

The three factors (classes) and some of elements of each factor are:

\begin{enumerate}

    \item knowledge factor: something the user knows (e.g. a password, Partial Password, pass phrase, or personal identification number (PIN), challenge response (the user must answer a question, or pattern), Security question

    \item ownership factor: something the user has (e.g. wrist band, ID card, security token, cell phone with built-in hardware token, software token, or cell phone holding a software token)

    \item inherence factor: something the user is or does (e.g. fingerprint, retinal pattern, DNA sequence (there are assorted definitions of what is sufficient), signature, face, voice, unique bio-electric signals, or other biometric identifier)

\end{enumerate}




