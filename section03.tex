%!TEX root =  main.tex

\chapter{Access Management Suite}

\section{Suite Overview}

Around 2010 Oracle corporation acquiree a dozen software vendors and
technology companies in the identity and access management market. Among
them were some niche players and also leaders, like Sun Microsystems
with several active products and a stable user base.

Developing a portfolio of twenty specialized systems to two or three
tightly integrated solutions definitely the overall plan, but not every
step is set straight on that path, looking at the footsteps we see
sometimes.

Next to the Identity Management Suite the Access Management Suite is the
second major product group, where the collected technologies deal with
run-time evaluation of user or system access.

The Access Manager itself is the heart of Suite, where other products or
new feature sets have been integrated or merged into.

\begin{itemize}
    \item Access Manager
    \item Identity Federation
    \item Security Token Service
    \item Mobile and Social
    \item Adaptive Authentication Service
    \item OAuth Services
    \item Identity Context
\end{itemize}



\section{Suite Applications}


\subsection{Access Management Access Manager}

As the central part of the suite, Access Manager delivers the
administrative user interface. All authentication mechanisms are defined
here and components and processes can be activated or cut-off in the
admin user interface. Business applications can be registered and
external data sources containing identity data may be connected.

Distributed components, the policy enforcement points, are located
closer to the protected business applications and request authentication
and authorization information from the central access server over
encrypted channels.


\subsection{Identity Federation}

While Access Manager takes care of a single domain (or multiple but
independent domains), Identity Federation helps to connect domains
(e.g.\ two organizations with integrated business activities).  In each
activity the roles and responsibilities are well defined. Identity
Federations supports SAML and OpenID\@.


\subsection{Security Token Service}

Token validation and generation to facilitate access to services across
security domains and beyond organizational boundaries. Essentially the
service acts as a trust-broker that receives and validates client
requests and generates appropriate tokens for a requested resource.


\subsection{Mobile and Social}

Mobile (for mobile devices): integrate iOS and Android mobile devices,
policy enforcement on those devices, single-sign-on for mobile apps and
browser based access, SDK on mobile devices.

Social (for social services): integrate authentication, policies for
several social services available on Internet.


\subsection{Adaptive Authentication Service}

Real-time and batch risk analytics to prevent fraud and misuse, risk
based authentication, additional authentication methods, like
One-Time-Passwords (OTP).


\subsection{OAuth Services}

OAuth 2.0 authentication client and server. Manage access control over
domain borders.


\subsection{Identity Context}

Enable dynamic adaption of permissions based on user related data, like
location, last transactions, third party informations, other assigned
permissions, etc.


This document covers topics from Access Management Access Manager only.

% vi:set lbr breakindent:
