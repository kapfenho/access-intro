%!TEX root =  main.tex

\chapter{Authentication}

\section{What's Authentication}

\emph{Authentication} is the act of confirming the truth of an attribute
of a single piece of data (a datum) claimed true by an entity. In
contrast with \emph{identification} which refers to the act of stating
or otherwise indicating a claim purportedly attesting to a person or
thing's identity, authentication is the process of actually confirming
that identity.


\section{Authentication Factors}

The ways in which someone may be authenticated fall into three
categories, based on what are known as the factors of authentication:
something the user \emph{knows}, something the user \emph{has}, and
something the user \emph{is}.

Each authentication factor covers a range of elements used to
authenticate or verify a person's identity prior to being granted
access, approving a transaction request, signing a document or other
work product, granting authority to others, and establishing a chain of
authority.


The three factor types and some of elements of each factor are:

\begin{description}

    \item[Knowledge Factor] Something the user knows --- e.g.\ a 
        password, Partial Password, pass phrase, or personal identification 
        number (PIN), challenge response (the user must answer a question, 
        or pattern), Security question

    \item[Ownership Factor] Something the user has --- e.g.\ wrist band, 
        ID card, security token, cell phone with built-in hardware token, 
        software token, or cell phone holding a software token

    \item[Inherence Factor] Something the user is or does --- e.g.\ 
        fingerprint, retinal pattern, DNA sequence (there are assorted 
        definitions of what is sufficient), signature, face, voice, 
        unique bio-electric signals, or other biometric identifier

\end{description}


For authentication to well-secured systems elements from at least two,
and preferably all three, factors should be verified.


\section{Authentication Methods}

What information the user is requested to enter during authentication
and how this information is passed to the service, this is defined as
authentication method.

Authentication methods are standardized, usually in RFC, to enable
interoperability and to ensure a certain system security among different
implementations.  Authentication is a critical process and is target to
all kinds of security attacks.

\begin{description}

    \item[User Passwords] \emph{Knowledge factor.}  The most common used
        method is also considered as the weakest and most problematic
        one.  Users seem to be very bad in \emph{choosing} password, in
        \emph{remembering} them and in \emph{changing} them
        periodically.\footnote{To increment an included number is not
        considered as proper password change.}
            
        Another risk is the transport of the real password over the
        wire, what is the typical implementation in web
        applications.\footnote{Mail clients offer alternative, standard
        transport methods for over a decade now.}
        
        On the other hand application developers seem to be good in
        creating unsecure implementations, especially in storing or
        dealing with passwords. Bottom line: if possible, prevent using
        passwords in general.
    
    \item[Kerberos] \emph{Knowledge and ownership factor.}  Kerberos is
        a distributed ticketing system, passing encrypted keys a
        credentials. Kerberos version 5 is included in every mainstream
        Unix/Linux/BSD system and Microsoft Windows as the default
        authentication method.
        
        Kerberos is the preferred method of authentication within LAN
        network, but it was not widely used on WAN because an additional
        communication port would have been necessary. With MS-KKDCP from
        Microsoft this seems to change, it has been adopted by many open
        source groups and vendors.
        
    \item[RSA Token] \emph{Knowledge and ownership factor.}  Proprietary
        implementation from RSA Security Inc., currently a division of
        EMC Corporation. Often used in Enterprise VPN solutions. RSA is
        known to distribute products with backdoor for security
        services.  Especially problematic are RSA libraries included in
        3rd party products (e.g.\ Dual\_EC\_DRBG in Windows, Java,
        OpenSSL).

        While its not recommended to use RSA Token, several alternatives
        do exist that are considered secure, some of them are internet
        standards.\footnote{RFC 1760 (S/KEY), RFC 2289 (OTP), RFC 4226
        (HOTP) and RFC 6238 (TOTP).} More user-friendly then hardware
        tokens are software versions for mobile devices, providing multi-
        service-provider support.

    \item[Client Certificates] \emph{Knowledge and ownership factor.}
        X.509 certificates are considered to be secure. Prerequisite is
        a private key infrastructure (PKI) with an online revocation
        service though. Client certificates are often used within an
        organization with an independent, self-signed certification
        authority. This requirement makes it difficult (or too
        expensive) for global usage. This might change in the near
        future, as changes in the CA business model are already taking
        place.

\end{description}


            
        

