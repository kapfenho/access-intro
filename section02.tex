%!TEX root =  main.tex

\section{Authentication}

\subsection{What's Authentication}

\emph{Authentication} is the act of confirming the truth of an attribute
of a single piece of data (a datum) claimed true by an entity. In
contrast with \emph{identification} which refers to the act of stating
or otherwise indicating a claim purportedly attesting to a person or
thing's identity, authentication is the process of actually confirming
that identity.


\subsection{Authentication Factors}

The ways in which someone may be authenticated fall into three
categories, based on what are known as the factors of authentication:
something the user \emph{knows}, something the user \emph{has}, and
something the user \emph{is}.

Each authentication factor covers a range of elements used to
authenticate or verify a person's identity prior to being granted
access, approving a transaction request, signing a document or other
work product, granting authority to others, and establishing a chain of
authority.


The three factor types and some of elements of each factor are:

\begin{description}

    \item \emph{knowledge factor}: something the user knows, e.g.\ a 
        password, Partial Password, pass phrase, or personal identification 
        number (PIN), challenge response (the user must answer a question, 
        or pattern), Security question

    \item \emph{ownership factor}: something the user has (e.g.\ wrist band, 
        ID card, security token, cell phone with built-in hardware token, 
        software token, or cell phone holding a software token)

    \item \emph{inherence factor}: something the user is or does (e.g.\ 
        fingerprint, retinal pattern, DNA sequence (there are assorted 
        definitions of what is sufficient), signature, face, voice, 
        unique bio-electric signals, or other biometric identifier)

\end{description}


For authentication to well-secured systems elements from at least two,
and preferably all three, factors should be verified.


\subsection{Authentication Methods}

TODO


